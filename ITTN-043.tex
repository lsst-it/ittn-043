\documentclass[PMO,authoryear,toc]{lsstdoc}
% lsstdoc documentation: https://lsst-texmf.lsst.io/lsstdoc.html
\input{meta}

% Package imports go here.

% Local commands go here.

%If you want glossaries
%\input{aglossary.tex}
%\makeglossaries

\title{Rubin Network Re-Engineering}

% Optional subtitle
% \setDocSubtitle{A subtitle}

\author{%
Cristian Silva
}

\setDocRef{ITTN-043}
\setDocUpstreamLocation{\url{https://github.com/lsst-it/ittn-043}}

\date{\vcsDate}

% Optional: name of the document's curator
% \setDocCurator{The Curator of this Document}

\setDocAbstract{%
The following document details the plan of Rubin IT to re-engineer the network upon the review done by the Network Review Committee. 
}

% Change history defined here.
% Order: oldest first.
% Fields: VERSION, DATE, DESCRIPTION, OWNER NAME.
% See LPM-51 for version number policy.
\setDocChangeRecord{%
  \addtohist{1}{YYYY-MM-DD}{Unreleased.}{Cristian Silva}
}


\begin{document}

% Create the title page.
\maketitle
% Frequently for a technote we do not want a title page  uncomment this to remove the title page and changelog.
% use \mkshorttitle to remove the extra pages

% ADD CONTENT HERE
% You can also use the \input command to include several content files.

\appendix
% Include all the relevant bib files.
% https://lsst-texmf.lsst.io/lsstdoc.html#bibliographies
\section{References} \label{sec:bib}
\renewcommand{\refname}{} % Suppress default Bibliography section
\bibliography{local,lsst,lsst-dm,refs_ads,refs,books}

% Make sure lsst-texmf/bin/generateAcronyms.py is in your path
\section{Acronyms} \label{sec:acronyms}
\addtocounter{table}{-1}
\begin{longtable}{p{0.145\textwidth}p{0.8\textwidth}}\hline
\textbf{Acronym} & \textbf{Description}  \\\hline

AI & Artificial Intelligence \\\hline
ALMA & Atacama Large Millimeter Array (ESO) \\\hline
BGP &  Border Gateway Protocol \\\hline
BSR & Business Systems Review \\\hline
CVE & Common Vulnerabilities and Exposures \\\hline
EOS & Engineering Operations Services \\\hline
ESO & European Southern Observatory \\\hline
FY21 & Financial Year 21 \\\hline
IT & Information Technology \\\hline
L2 & Lens 2 \\\hline
LAN & Local Area Network \\\hline
LHN & long haul network \\\hline
NCSA & National Center for Supercomputing Applications \\\hline
NOIRLab & NSF's National Optical-Infrared Astronomy Research Laboratory; \url{https://nationalastro.org} \\\hline
NSF & National Science Foundation \\\hline
OS & Operating System \\\hline
PMO & Project Management Office \\\hline
TAC & Arista Global Technical Assistance Center \\\hline
US & United States \\\hline
\end{longtable}

% If you want glossary uncomment below -- comment out the two lines above
%\printglossaries





\end{document}

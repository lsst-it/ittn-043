\section{Conclusion}

All vendors provide quality hardware and meet the technical requirements of Rubin IT's design. However, support, licensing, and experience with the products are essential factors that have to be considered in the final recommendation. 

\subsection{Support}
The team doesn't have good experiences with Cisco support. It has been challenging to talk with a Cisco Engineer, and going through the vendor has sometimes complicated more the issues. It is clear that after speaking with a Cisco Engineer, the flow of the communication improves, and problems are solved in a reasonable time. 

There's no experience with Arista and Juniper support, however, during the evaluation of the vendors, both provided access to virtual labs and engineers to discuss our design. It is the expectation that both vendors will behave similarly to our experience with Cisco Engineers. 

\subsection{License}
Cisco has shifted its licensing scheme to a subscription-based. This means that software licenses have to be renewed every few years, increasing the cost of equipment ownership and creating uncertainty for budget planning. There are also changes in features provided, hence there's no clarity if the current license tier will cover the needs of the network.

Arista and Juniper provide perpetual licensing.

\subsection{Experience}
Rubin IT and Noirlab engineers have plenty of experience with Cisco devices and little experience with Arista; however, given the similarity of the operating systems of both vendors, the Arista learning curve is relatively small. On the other hand, Juniper has an entirely different configuration structure based on commits, which could be beneficial in some cases but represents a much higher learning curve. 

\subsection{Recommendation}

Given this evidence, the best path is to migrate the Cisco ACI platform to a non-vendor locking technology based on Arista switches. 

Arista represents a small learning curve for Rubin and Noirlab engineers, provides all the features needed for the project, is competitively priced, and offers a better cost of ownership given its perpetual licensing model. 
The technology will be not be bound to a vendor, so unlike the current state, there will be a mix of vendors in the network, allowing Rubin to select the best available solution regardless of the vendor. 

The complexity of the network will be severely reduced, facilitating the troubleshooting of problems and increasing the time on sky.

Arista also has considerably fewer vulnerabilities reported in the last seven years. Cisco accounts for 2500 vulnerabilities, Juniper 300 and Arista 14. This is an essential factor to consider because each vulnerability, depending on its severity, could mean patching to the network equipment and potential interruptions or outages. It also brings up the topic of the quality of each vendor's software; it is clear that for Arista, security is an important.

The migration to Arista still needs to be analyzed and scheduled and will be laid out in a separate document. 